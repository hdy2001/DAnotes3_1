% !Mode:: "TeX:UTF-8"
\documentclass[UTF8,a4paper,landscape,8pt]{paper}
\usepackage{ctex}
\usepackage{amsmath}
\usepackage{pdfpages}
\usepackage{graphicx}
\usepackage{wrapfig}
\usepackage{listings}
\usepackage{xcolor}
\usepackage{amssymb}
\usepackage{bm}
\usepackage{enumitem}
\usepackage{multicol}
\usepackage{float}
\usepackage{cuted}
\usepackage{titlesec}
\usepackage{fancyhdr}
\usepackage[landscape]{geometry}
\usepackage[figuresright]{rotating}
\linespread{0.5}
\geometry{left =0cm,right = 0.1cm,top = 0.1cm, bottom = 0.1cm}
\pagestyle{fancy}
\renewcommand{\subsection}[1]{{\small\textbf{\underline{#1}}}\\ }
\renewcommand{\section}[1]{{\normalsize\textbf{\emph{#1}}}\\ }
\newcommand{\List}[1]{\begin{itemize}[fullwidth,itemindent=0em] #1 \end{itemize}}
\begin{document}\footnotesize
\begin{multicols}{4}
\section{数值分析和科学计算引论}
    \List{
    \item {$x^*$是$x$的近似值,$e^* = x^* - x$称为绝对误差,
    $e^*_r = e^* / x $为相对误差,上限为绝对误差限和相对误差限}
    \item {避免产生误差危害的方法:先小后大;避免相近相减;避免交错级数;避免被除数远大于除数;松弛技术}
    }
\section{插值法}
    \subsection{Lagrange插值法}
    \List{
        \item {$n$次Lagrange插值基函数$l_j(x)$满足
        $l_j(x_k) = \delta_{kj}$}
        \item {$l_k(x) =\prod_{j\ne k} (x-x_j) / \prod_{j\ne k} (x_k-x_j)$}
        \item {$\omega_{n+1}(x) = \displaystyle{\prod_{j=0}^n(x-x_j)}, \omega'(x_k)=\displaystyle{\prod_{j=0,j\ne k}^n(x_k-x_j)}$}
        \item {$l_k(x) = \frac{\omega_{n+1}(x)}{(x-x_k)\omega_{n+1}'(x_k)},L_n(x)=\sum y_kl_k(x)$}
        \item {$R_n(x) = \frac{f^{(n+1)}(\xi)}{(n+1)!}\omega_{n+1}(x) \le \frac{M}{(n+1)!}
        \max\{\omega_{n+1}\}$\\ 证明$R_n(x) = f(x)-L_n(x) = K(x)\prod(x-x_i)$,$\phi(t) = K(x_0)\prod(t-x_i) -f(t) + L_n(t)$,在节点和$x_0$上为零,使用罗尔定理}
        \item {插值基函数性质:$\sum x_i^k l_i(x) = x^k, \sum (x_i-x)^k l_i(x) = 0, \sum l_i(x) = 1$}
		\item {计算代价:总体$\mathcal O(n^2)$,单步:$\mathcal O(n^2)$}
    }
    \subsection{Newton插值法}
    \List{
        \item {一阶均差:$f[x_0,x_k] = \frac{f[x_k] - f[x_0]}{x_k-x_0}$}
        \item {二阶均差:$f[x_0,x_1,x_k] = \frac{f[x_0,x_k] - f[x_0,x_1]}{x_k-x_1}$}
        \item {k阶:$f[x_0,\cdots, x_k] = \frac{f[x_0,\cdots,x_{k-2},x_k] - f[x_0,\cdots, x_{k-1}]}
        {x_k-x_{k-1}}$}
        \item {$f[x_0,\cdots,x_k] = \sum^k_{j=0}[\frac{f(x_j)}{\prod^k_{i\ne j,i=0}(x_j-x_i)}]$}
        \item {轮转对称性:$f[x_0,\cdots,x_k] = f[Q[x_0,\cdots,x_k]]$}
        \item {若$f(x)$在$[a,b]$上存在$n$阶导数,$f[x_0,\cdots,x_n]=\frac{f^{(n)}(\xi)}{n!}$}
        \item {插值公式$P_n(x) = P_{n-1}(x) + f[x_0,\cdots,x_n] \omega_n(x)$}
        \item {余项$R_n(x) = f[x,x_0,\cdots,x_n]\omega_{n+1}(x) = \frac{f^{(n+1)}(\xi)}{(n+1)!}$}
        \item {一阶向前差分$\Delta f_k = f_{k+1} - f_k$,向后差分$\nabla f_k = f_k - f_{k-1}$,中心差分$\delta f_k = f_{k+\frac{1}{2}} - f_{k-\frac{1}{2}}$}
        \item {位移因子$Ef_k=f_{k+1}, \Delta = E - I, \nabla = I - E^{-1}$,可以二项展开,$(a+b)^n = \sum_{i=0}^n a^i b^{n-i} \frac{n!}{i!(n-1)!}$}
        \item {Newton等间隔(前插)公式,下面$h$为插值节点间隔}
        \item {$P(x_0+th) = \sum_{i=0}^n\frac{\prod_{j=1}^i(t-j+1)}{i!}\Delta^if_0 = f_0 + t\Delta f_0 + \frac{t(t-1)}{2!}\Delta^2f_0 + \cdots + \frac{t(t-1)\cdots(t-n+1)}{n!}\Delta^nf_0$ }
        \item {余项 $R_n(x) = t(t-1)\cdots(t-n)h^{n+1}f^{(n+1)}(\xi)/(n+1)!$,t是变量}
		\item {计算代价:总体$\mathcal O(n^2)$,单步:$\mathcal O(n)$}
        }
    \subsection{埃尔米特插值:给定节点和节点上的若干导数}
    \List{
        \item {$R_n(x) = \frac{f^{(m+1)}(\xi)}{(m+1)!}\prod(x-x_i)^{\alpha_i}$}
        \item {$\alpha_i$ 为每个节点上的条件个数,$m = \sum \alpha_i - 1$}
    }
    \subsection{分段低次插值与三次样条插值,$h$为区间长度}
    \List{
        \item {分段线性插值:$|R_n(x)| \le \frac{h^2}{8}M_2$。三次埃尔米特插值:$|R_n(x)| \le \frac{h^4}{384}M_4$(每个点多给出导数,单一区间)}
        \item {三次样条插值:$0\rightarrow n$共n+1个点,4n个待定参数。插值函数值,一阶、二阶导数连续($3n-3$个条件),每个点的函数值($n + 1$个条件),两个边界条件,确定。}
        \item {常见边界条件:1. 给定两端一阶导数;2. 给定两端二阶导数,为零:自然边界条件;3. $x_0,x_n$处二阶导数,一阶导数相等(再加上原来相等,为周期函数) }
        \item {三次样条误差界:$R^{(k)}(x) = M_4 h^{4-k} C_k$对应各阶导数的误差,$C_0 = \frac{5}{384},C_1 = \frac{1}{24},C_2=\frac{3}{8}$}
    }
\section{函数逼近}
    \List{
        \item {最小零偏差:函数范数最小}
        \item {向量范数:$\|x\| \ge 0, \|x\| =0 \iff x = 0, \|\alpha x\| = |\alpha|\|x\|, \|x+y\| \le \|x\| + \|y\|$ 当且仅当线性相关取等号}
        \item {内积空间:$|(u,v)|^2 \le (u,u)(v,v)$,当且仅当线性相关取等号,$|(u_i,u_j)_{i,j}| \ne 0 \iff $ 向量组$u_*$线性无关 }
        \item {函数范数$\|f\|_\infty = \max\{|f|\}, \|f\|_1 = \int_a^b|f|\mathrm d x, \|f\|_2 = \sqrt{\int_a^bf^2\mathrm d x}$,对应函数逼近分别称为(一致)逼近,平方逼近}
        \item {函数带权内积$<f,g> = \int_a^bfg\rho\mathrm dx,\rho(x) \ge 0$且连续}
    }
    \subsection{正交多项式性质}
    \List{
        \item {任何$P(x)\in H_n$可表示为$\{\phi\}^n_0$的线性组合}
        \item {$\phi_n(x)$与任何次数小于$n$的多项式$P(x)\in H_{n-1}$正交}
        \item {$\{\phi_x(x)\}$在$[a,b]$上正交,$\phi_0 = 1,\phi_{-1} = 0 \Rightarrow$ \\ $ \phi_{n+1} = (x-\alpha_n)\phi_n(x)-\beta_n\phi_{n-1}x$。证明思路:$x\phi_n$正交化,并从积分上分析$(x\phi_n,\phi_{n-1}) = (\phi_n,x\phi_{n-1})=(\phi_n,\phi_n),(x\phi_n,\phi_{i}) = (\phi_n,x\phi_{i})=0,i < n-1$}
        \item {$\{\phi\}^n_0$正交,则$\phi_n$在区间上有$n$个不同的零点。证明思路将所有零点补成偶数重,则内积大于零(正定),反证法。}
    }
    \subsection{最佳平方逼近,Legendre多项式:$\widetilde {P_n}$是首一的}
    \List{
        \item {带权$\rho = 1,[-1,1]$正交,$P_0=1;P_n=\frac{\mathrm d^n}{\mathrm dx^n} \frac{(x^2-1)^n}{2^nn!};$ \\ $ (P_n,P_m) = \frac{2}{2n+1}\delta_{m,n}; P_n(-x) = -1^nP_n(x); \widetilde {P_n} = \frac{n!}{(2n)!}\frac{\mathrm d^n}{\mathrm dx^n}(x^2-1)^n; (n+1)P_{n+1} = (2n+1)xP_n - nP_{n-1}$}
        \item {$P_2 = \frac{3x^2-1}{2},P_3 = \frac{5x^3-3x}{2},P_4=\frac{35x^4-30x^2+3}{8}$}
        \item {正交函数最佳平方逼近,$a_k=\frac{(f,\phi_k)}{(\phi_k,\phi_k)},f\approx P^* = \sum a_k\phi_k$,余项最低次为$n+1$。$\widetilde {P_n}$是$\widetilde {H_n}$中平方范数最低的}
        \item {对线性无关非正交多项式函数组做逼近:求$n$元方程$\sum_j(\psi_i,\psi_j)a_j = (\psi_i,f), f \approx S = \sum a_i\psi_i$,最小偏差表达式:$(f,f) - (S,f)$}
    }
    \subsection{最佳一致逼近,Chebyshev多项式,$\widetilde{T_n}$是首一的}
    \List{
        \item {带权$\rho = \frac{1}{\sqrt{1-x^2}}$正交;$T_n(x) = \cos(n\arccos(x))$,$x \in[-1,1],T_{n+1} = 2xT_n -T_{n-1},(T_m,T_n) = \frac{\pi}{2}\delta_{m,n}(m \ne 0) = \pi (m = n = 0);P_n(-x) = -1^nP_n(x)$;零点位置;$x_k = \cos\frac{2k-1}{2n}\pi,k = 1,\cdots,n$;$T_n$首项系数$2^{n-1}$}
        \item {$\forall P\in\widetilde{H_n},\|T_n\|_\infty\le \|P\|_\infty$,求取低一次的一致逼近多项式$P* = P - a_1\widetilde{T_n}$,$a_1$是$P$首项系数。}
        \item {Chebyshev多项式插值:采用$T_{n+1}$零点作为$n$次插值多项式的节点,$\|f-L_n\|_\infty \le \frac{M_{n+1}}{2^n(n+1)!}$}
        \item {最佳一致逼近:$\Delta(f,P_n) = \|f-P_n\|_\infty$偏差点$|f-P_n|_{x_0} = \Delta(f,P_n)$,正偏差点$f-P_n = -\Delta(f,P_n)$,负偏差点$f-P_n = \Delta(f,P_n)$}
        \item {最佳一致逼近多项式总存在。Chebyshev定理:$P^*_n$是最佳一致逼近,则在$f$上至少有$n+2$个正负相间的偏差点:当不合理的时候可能有更多。}
        \item {$T_2 = 2x^2-1,T_3 = 4x^3 -3x, T_4 = 8x^4 - 8x^2 + 1,T_5 = 16x^5-20x^3+5x,T_6 = 32x^6-48x^4+18x^2-1$ }
        \item {$\widetilde{T_n}$是首一同次多项式中无穷范数最低的}
    }
    \subsection{曲线拟合和最小二乘}  
    \List{
        \item {离散空间带权$\omega$内积:$(f,\phi) = \sum \omega(x_i)\phi(x_i)f(x_i)$}
        \item {函数组在区间线性相关 $\Rightarrow$ 函数在若干点上线性相关(存在某个线性组合使得在某些点上线性组合为0}
        \item {$H_n$函数在$n+1$个点上线性相关 $\Rightarrow$ 函数组线性相关}
    }
\section{数值积分与数值微分}
    \List{
        \item {梯形公式:$\int_a^b f\mathrm dx \approx \frac{b-a}{2}[f(a)+f(b)]$,中矩形公式:$I \approx (b-a)f(\frac{a+b}{2})$,机械求积公式:$I = \sum A_kf(x_k)$,$A_k$成为求积系数或权,仅与$x_k$有关。$\sum A_k = 1$}
        \item {代数精度:求积公式对不超过$m$次多项式均成立,对$m+1$次不成立,则称$m$次代数精度,常用非线性相关多项式族$x^n$进行测试}
        \item {机械求积公式选取$x_0,\cdots,x_m$共$m+1$个点至少保证$m$次代数精度(强行解方程可得),此时系数唯一}
        \item {插值型求积公式:$\int L_n = \int (\sum f_kl_k) = \sum f_k \int l_k$\\ $A_k = \int l_k$,与$n$次代数精度等价($n$次代数精度$\rightarrow A_k$唯一 $\rightarrow I = \sum A_kf_k \rightarrow \int l_k = \sum A_k f_k = A_k$(利用$l_k(x_j) = \delta_{j,k}$}
        \item {余项$R[f] = \int R_n = \int \frac{f^{n+1}}{(n+1)!}\omega_{n+1} \overset{\text{中值定理}}{=} Kf^{n+1}$。推广对于$m$次代数精度的求积式,$R[f] = Kf^{(m+1)},K$由对$f=x^{m+1}$求积($f^{(m+1)}=\mathrm{const.}$)定量求得(形式必为$R[f] = k(b-a)^{m+2}M_{m+1}$,因此可以在$[0,1]$下计算器求解)。对梯形公式:$m=1,R[f] \le \frac{(b-a)^3}{12}M_2$,中矩形公式:$m=1,R[f] \le \frac{(b-a)^3}{24}M_2$}
        \item {收敛性:$h$区间足够小趋近真值。稳定性:$f$有误差$\widetilde f$误差不放大$\rightarrow \forall k,A_k >0$,插值型积分公式不一定稳定}
        \item{Newton-柯特斯公式:积分区间等距取点的插值型机械求积公式,辛普森公式($n=2$):$S = \frac{b-a}{6}[f(a) + 4 f(\frac{a+b}{2}) + f(b)],R[f] < \frac{(b-a)^5}{180*16}M_4$,柯特斯公式($n=4$):$S = \frac{b-a}{90}[7f(x_0) + 32f(x_1) + 12f(x_2) + 32 f(x_3) + 7 f(x_4)], R[f] < \frac{(b-a)^7}{945*2048}M_6$,$n=8$,积分公式不稳定,出现误差放大,放大倍数为$\sum |A_k| > \sum A_k = 1$,$n$为偶数时,代数精度为$n+1$,为奇数时为$n$(采用中值定理时通过埃尔米特插值方法保证其他项不变号)}
    }
    \subsection{复合求积公式和Romberg 求积公式}
    \List{
        \item {复合梯形 $T_n = \frac{h}{2}[f(a) + 2 \sum f(x_k) + f(b)], R_n[f] = \frac{b-a}{12}h^2M_2$}
        \item {复合Simpson $S_n = \frac{h}{6}[f(a) + 4 \sum f(x_{k+\frac12}) + 2\sum f(x_k) + f(b)],R_n[f] = -\frac{b-a}{180}(\frac{h}{2})^4M_4$}
        \item {梯形公式递推:$T_{2n} = \frac{T_n}{2} + \frac{h}{2}\sum f(x_k+\frac12)$}
        \item {Richardson外推加速法:$T(h) = I + \alpha h^2 + \cdots $,利用$T(h),T(\frac h2)$消除$h^2$项得到$S(h)$。逐渐外推加速。递推公式$T_m(h) = \frac{4^m}{4^m-1}T_{m-1}(\frac h2) - \frac {T_{m-1}(h)}{4^m-1}$}
        \item {Romberg 求积算法:1. 取$k=0,h=b-a,T_0=\frac h2[f(a)+f(b)]$ 2. 加速求$T$,包括一个梯形值$T_0(\frac{b-a}{2^k})$,到达给定精度结束。$h$越小代表复化程度越高,阶次越高代表求积越复杂。} 
    }
    \subsection{Gauss求积公式}
    \List{
        \item {求取$A_k,x_k$使得机械求积有$2n+1$次精度(正好全部满足)。充要条件:$x_k$是$n+1$次带权正交多项式的零点:证明思路:前$n$次说明$A_k$取值为插值型,后$n+1 - 2n+1 $次对应导数为$x \cdots x_n$带权内积为0,说明高一次正交。积分权重全部是正的$\int l_k^2$可以准确得到,即$\sum A_il_k^2 = A_k >0$}
        \item {Gauss-Legendre求积:Legendre多项式零点。$\int^1_{-1}f = 2f(0) = f(-\frac{1}{\sqrt{3}}) + f(-\frac{1}{\sqrt{3}}) = \frac59f(-\frac{\sqrt{15}}5) + \frac89f(0) + \frac59f(\frac{\sqrt{15}}5), R_n[f] = \frac{2^{2n+3}[(n+1)!]^4}{(2n+3)[(2n+2)!]^3}M_{2n+2}$}
        \item {Gauss-Chebyshev求积:$\int_{-1}^1 \frac f{\sqrt{1-x^2}} = \frac \pi n\sum f(x_k), x_k = \cos\frac{2k-1}{2n}\pi, 1 \le k \le n$}
    }
    \subsection{数值微分}
    \List{
        \item {中点方法 $G(h) = \frac{f(a+h) - f(a-h)}{2h}, |f'(a) -G(h)| \le \frac{h^2}{6}M_3$,可以外推加速,但是要对余项进行分析。$G_m(h) = \frac{4^mG_{m-1}(\frac h2) - G_{m-1}(h)}{4^m-1}$}
        \item {插值方法:$f' \approx P'$,求导数的点就在某个节点上,$R = R_n' \le \frac{M_{n+1}}{(n+1)!}\omega_{n+1}'$,其中一点方法为直接求斜率,余项$\frac h2 f''$}
    }
\section{非线性方程与方程组的数值解法}
    \List{
        \item {不动点迭代法:$f(x) = 0 \rightarrow \phi(x) = x$迭代。不动点存在条件为$\exists a,b\forall x \in [a,b]. a \le \phi(x) \le b, \exists L < 1, s.t. |\phi(x)-\phi(y)| \le L|x-y|, |\phi(x)'| < 1$}
        \item {序列误差估计$x_k-x^* \le \frac {L^k}{1-L}|x_1-x_0|, e_k = x_k - x*, \lim_{k\to\infty} \frac{e_{k+1}}{e_k^p} = C \ne 0$则称$p$阶收敛}
        \item {迭代法局部收敛:存在邻域$R$使得$|\phi(x)| < 1$,需要选取初始点}
    }
    \subsection{Newton法及其衍生方法}
    \List{
        \item {Newton法:$\psi(x) =x - \frac{f(x)}{f'(x)}$,平方收敛速度}
        \item {$\psi'(x) = \frac{ff''}{(f')^2},\phi''(x^*) = \frac{f''(x^*)}{f'(x^*)}$}
        \item {Newton下山法:下山条件$|f_{k+1} < f_k|, \psi(x) = x-\lambda \frac f{f'}$从$\lambda=1$开始试算,直到满足下山条件}
        \item {简化Newton法:$\psi = x - \frac{f(x)}{f'(x_0)}$,线性速度}
        \item {出现$m$重根时,普通Newton法变为一次收敛,$\psi = x -\frac f {mf'},\psi = x - \frac{ff'}{(f')^2 - ff''}$二阶收敛。}
        \item {弦截法:$x_{k+1} = x_k -\frac{f(x_k)(x_k - x_{k-1})}{f(x_k) - f(x_{k-1})},\frac{1+\sqrt 5}2$阶收敛,抛物线法:1.840阶收敛}
    }
\section{常微分方程初值问题的数值解法}
    \subsection{线性单步法}
    \List{
        \item {满足L.条件的微分方程误差传播$|y(x,x_0) - y(x,\widetilde{x_0}| \le \exp(L|x_0 - \widetilde{x_0}|)|x_0 - \widetilde(x_0)|$,数值计算过程中等距取点迭代计算,$y(x_{n+1}) = y(x_n) + h\psi, y'=f(x,y)$}
        \item {局部截断误差$y -\widetilde y =\mathcal O(h^{p+1})$,称为$p$阶精度,其$h^{p+1}$对应主项为阶段误差主项}
        \item {欧拉法:$\psi =y'(x_n)$,1阶局部误差精度;向后欧拉法(隐式):$\psi = y'(x_{n+1})$,一阶局部误差精度;梯形方法(隐式):$\psi = \frac{y'(x_n) + y'(x_{n+1})}{2}$2阶局部误差精度;改进欧拉法:$\psi = f(x_n,y_n) + f(x_{n+1},\overline{y_{n+1}}),\overline{y_{n+1}} = y_n + hf(x_n,y_n)$2阶局部误差精度。隐式方法需要通过迭代确定未知值}
    }
    \subsection{线性多步法}
    \List{
        \item{$G(x) = f(x,y(x)),y(x_{n+1}) = y(x_n) + \int_{x_n}^{x^{n+1}} G(x) $ 积分过程,$G(x)$通过前面得到的点插值得到}
        \item {显性公式(用于预测)一般形式:$\psi = \sum\beta_{mi}f_{n-i}$\\ $\beta_{mi}$是标准化$m$次插值基函数积分值}
        \item {隐性公式(用于矫正:将采样点包括了即将计算的点)$\psi = \sum\beta_{mi}f_{n-i+1}$}
        \item {局部截断误差相当于积分误差}
    }
\section{解线性方程组的直接解法}
    \List{
        \item {L的逆矩阵仍是L矩阵,因此LU分解唯一}
		\item {直接方法:适用于低阶矩阵,迭代方法:适用于高阶矩阵}
        \item {矩阵$\bm{A}$的特征值组合成为$\bm{A}$的谱$\sigma(\bm{A})$,特征值绝对值的最大值称为其谱半径$\rho(\bm{A})$}
        \item {矩阵范数:基本要求:$\|\bm A\|\ge 0,\|\bm A\| = 0 \iff A = 0,\|c\bm A\| = |c|\|\bm A\|, \|\bm A + \bm B \| = \|\bm A\| + \| \bm B\|, \|\bm A\bm B\| = \|\bm A\|\|\bm B\|$}
        \item {算子范数:结合向量范数给出,$\|\bm A\|_v \triangleq \max \frac {\|\bm A x\|_v}{\|x\|_v}$}
        \item {$\|\bm A\|_\infty= \max_{1\le i \le n}\sum_j|a_{ij}|$,最大的行的绝对值的和}
        \item {$\|\bm A\|_1= \max_{1\le j \le n}\sum_i|a_{ij}|$,最大的列的绝对值的和}
        \item {$\|\bm A\|_2= \sqrt{\lambda_{\max}(\bm A^T\bm A)}$,最大的特征值的平方根}
        \item {向量范数连续。$\forall s,t \exists c_1,c_2, c_1\|x\|_s \le \|x\|_t \le c_2\|x\|_s$}
        \item {$\rho(\bm A) \le \|\bm A\|. \forall \epsilon, \exists (\|\cdot\|_{\epsilon}), \|\bm A\|_{\epsilon} \le \rho(\bm A) + \epsilon$,证明谱半径关系可以通过范数来计算}
        \item {$\|\bm Ax\|_v \le \|\bm A\|_v\|x\|_v.\bm A^T = A\Rightarrow \|\bm A\|_2 = \rho(\bm A)$}
        \item {$\|\bm B\| \le 1 \Rightarrow |\bm I\pm \bm B| \ne 0, (\bm I - \bm B)^{-1} = \bm I + \bm B(\bm I - \bm B)^{-1} \Rightarrow \|(\bm I \pm \bm B)^{-1}\| \le \frac 1{1-\|\bm B\|}$}
        \item {矩阵条件数:$\mathrm{cond}(\bm A)_v = \|\bm A^{-1}\|_v\|\bm A\| \ge \|\bm I\|_v = 1$}
        \item {$\mathrm{cond}(c\bm A)_v = \mathrm{cond}(\bm A)_v; \bm A = \bm A^T \rightarrow \mathrm{cond}(A)_2=|\lambda_1/\lambda_n|;\bm A\bm A^T = \bm I \rightarrow \mathrm{cond}(\bm A)_2 = 1,\mathrm{cond}(\bm A \bm B)=\mathrm{cond}(\bm B \bm A) = \mathrm{cond}(\bm B) $}
    }
    \subsection{LU分解和P-LU分解}
    \List{
        \item {手算过程中常常采用初等行变换方法得到$U,L^{-1}$,需要对右侧矩阵求逆:对角线为1的三角阵求逆即是将其他元素取相反数}
        \item {LU 分解唯一,PLU分解不一定唯一,LU分解要求各阶顺序主子式非零,P-LU分解要求矩阵非奇异}
        \item {回代消元法计算代价$\frac{n^3}{3}-n^2-\frac n3$次乘除,$\frac{n^3}{3}-n^2-\frac n3$次加减}
        \item {P-LU 分解复杂度$\mathcal O(\frac{n^3}3)$,平方根法计算复杂度$\mathcal O(\frac{n^3}6)$}
        \item {平方根法:$\bm A = \bm A^T \Rightarrow \bm A = \bm L\bm D \bm L_T = \bm L_1 \bm L_1^T$}
    }
    \subsection{误差分析}
    \List{
        \item {$\bm A(x+\delta x) = (b+\delta b) \Rightarrow \frac{\|\delta x\|}{\|x\|}\le \mathrm{cond}(\bm A) \frac{\|\delta b\|}{\|b\|}$}
        \item {$(\bm A + \delta\bm A)(x+\delta x) = b \Rightarrow \frac{\|\delta x\|}{\|x\|}\le \frac{\mathrm{cond}(\bm A) \frac{\| \delta \bm A\|}{\|\bm A\|}}{1-\mathrm{cond}(\bm A)\frac{\| \delta \bm A\|}{\|\bm A\|}}$}
        \item {近似:$(\bm A + \delta\bm A)(x+\delta x) = (b+\delta b) \Rightarrow \frac{\|\delta x\|}{\|x\|}\le \mathrm{cond}(\bm A) (\frac{\|\delta b\|}{\|b\|} + \frac{ \|\delta \bm A\|}{\|\bm A\|})$}
        \item {事后估计法:得到$\widetilde x$,$\frac{\|\widetilde x - x^*\|}{\|x^*\|} \le \mathrm{cond}(\bm A)\frac{\|b - \bm A \widetilde x\|}{\|b\|}$}
    }
\section{解线性方程组的迭代法}
    \List{
        \item {矩阵收敛:$\lim_{k\to\infty}\bm A_k = \bm A\iff\lim_{k\to\infty}\|\bm A_k-\bm A\| = 0,\lim_{k\to\infty} \bm A_k =0 \iff \lim_{k\to\infty} \bm A_kx=0, \forall x\in \mathbb R^n$}
        \item {迭代法基本形式:$x = \bm B x + f \iff \bm A x = b$,收敛条件$\rho(\bm B) < 1$}
        \item {可以采用$\rho(\bm B) \le \|\bm B\|_v <1$的任意从属范数来证明,$\lim_{k\to\infty}\|\bm B^k\|^{1/k} = \rho (\bm B)$}
        \item {$\|\bm B\|_v = q < 1 \Rightarrow \|x^*-x^{(k)}\|_v \le q^k \|x^*-x^0\|_v$\\ $\|x^*-x^{(k)}\|_v \le \frac{q\|x^{(k)}-x^{(k-1)}\|_v}{1-q} \le \frac{q^k\|x^{1}-x^{0}\|_v}{1-q} $\\ 如果精度取到$\sigma = 10^{-s}$迭代次数$K \ge \frac{s\ln10}{-\ln\rho (\bm B)}$}
        \item {平均收敛速度$R_k(\bm B) = -\ln(\|\bm B\|_v^{1/k})$,渐进收敛速度$R(\bm B) = -\ln \rho (\bm B) = \lim_{k\to\infty}R_k(\bm B)$}
    }
    \subsection{Jacobi和Gauss-Seidel迭代法:$\bm A = \bm D - \bm L - \bm U$}
    \List{
        \item {Jacobi迭代法:$\bm Dx = (\bm L + \bm U)x + b\Rightarrow\\\bm  B = \bm D^{-1}(\bm L + \bm U) = \bm I - \bm D^{-1}\bm A = \bm J$}
        \item {Gauss-Seidel迭代法:$(\bm D - \bm L) x = \bm U x + b \Rightarrow\\\bm  B = (\bm D - \bm L)^{-1}\bm U = \bm I - (\bm D - \bm L)^{-1}\bm A = \bm G$}
        \item {收敛性:$\rho (\bm J) <1,\rho(\bm G) <1$}
        \item {严格对角占优矩阵:对角线的绝对值大于同行其他元素绝对值的和,弱对角占优矩阵:大于等于}
        \item {严格对角占优矩阵和不可约弱对角占优矩阵Jacobi方法和Gauss-Seidel方法收敛(使用无穷范数代替谱半径证明)}
    }
    \subsection{逐次超松弛迭代法(SOR)}
    \List{
        \item {对Gauss-Seidel方法加速:$x^{(k+1)} = \bm D^{-1}[b - \bm Lx^{(k+1)} - \bm Ux^{(k)}] = x^{(k)} + \bm D^{(-1)}[b - \bm Lx^{(k+1)} - (\bm D + \bm U)x^{(k)}]$,通过对G-S的$x^{(k+1)}$和$x^{(k)}$加权得到下一个值}
        \item {$x^{(k+1)} = \omega x^{(k+1)} + (1 - \omega x^{(k)}) = \bm L_\omega x^{(k)} + f, \bm L_\omega = (\bm D + \omega \bm L)^{-1}[(1-\omega)\bm D -\omega \bm U]$}
        \item {收敛必要条件:$0 < \omega < 2$,证明思路:$\rho(\bm B) < 1 \Rightarrow |\mathrm{det}\bm B| < 1$}
        \item {$\bm A$正定对称且$0 < \omega < 2$,则SOR收敛}
        \item {舍入误差分析:$\|\delta_{k+1}\| \le \|\bm B\| \|\delta_k\| + \|\Delta\|$,$\|\Delta\|$为存储误差}
    }
\section{矩阵的特征值计算}
    \List{
        \item {$\bm A$的每个特征值必然在下面的某个圆盘之中:$|\lambda -a_{ii}| \le r_i =\sum_{i\ne j}a_{ij}$。换言之,$\bm A$的特征值在上述圆盘的并集中}
        \item {上述圆盘如果存在$m$个圆盘孤立于其他的圆盘,则必然这个区间内有$m$个特征值,特殊地,如果有一个孤立的圆盘,则其中必然仅有一个特征值,如果$A\in\mathbb R^{n\times n}$,则此时该特征值必然为实数}
        \item {$\mu =\lambda(\bm A + \delta(\bm A))\in \mathbb R^{n\times n}, \bm P^{-1}\bm A\bm P = \bm D$则$\min_{\lambda\in\sigma(\bm A)}|\lambda-\mu| \le \mathrm{cond}(\bm P)_p\|\delta(\bm A)\|_p$}
    }
    \subsection{幂法和反幂法}
    \List{
        \item {幂法:迭代$v^{k+1} = \bm Av^k$之后得到$\lim_{k\to\infty}\frac{v_{k+1}}{v_k} = \lambda_1$}
        \item {幂法计算流程:随机取$u_0\rightarrow$计算$v_1 = \bm Au_0,u_1 = \frac{v_1}{\max(v_1)}$不断计算,最终$\lambda_1 = \frac{v_{k+1}}{u_k} = \lim_{k\to\infty}\max(v_k)$,收敛速度$|\frac{\lambda_2}{\lambda_1}|$确定}
        \item {对于重根,幂法最终得到的特征值是正确的,对应的特征向量$u_k$可能随初始值选取的不同出现不同结果}
        \item {问题:只能求解最大的。原点平移方式$\bm A' = \bm A - c\bm I$可以改进速度,并求取最大的和最小的特征值}
        \item {Rayleigh商加速:求取$\frac{(\bm Au_k,u_k)}{(u_k,u_k)}$来获得最后的特征值,有更高的精度(余项$\mathcal O(\frac {\lambda_2}{\lambda_1}^2)$,相比于之前的速度高一次)}
        \item {反幂法:计算按模最小的特征值和对应的特征向量,在已知近似特征值的情况下可以通过此方法得到有效的精确值}
        \item {迭代方法:$u_0$随机,$v_k = \bm A^{-1}u_{k-1},u_k = \frac{v_k}{\max v_k}$\\ $\frac 1{\lambda_n} = \lim_{k\to\infty}\max v_k$}
        \item {收敛速度:$|\frac{\lambda_n}{\lambda_{n-1}}|$,原点平移之后为$|\frac{\lambda_n - p}{\lambda_{n-1} - p}|$}
        \item {Rayleigh商加速仍然可以提高最后一次计算的精度,一般事先现将$\bm A$LU分解方便之后求方程}
        }
        \subsection{正交变换和矩阵分解,QR方法}
        \List{
            \item{Householder 变换:$\omega^T\omega = 1, \bm H(\omega) = \bm I -2\omega\omega^T$构成反射矩阵,正交对称,$\bm H = \bm H^T = \bm H^{-1}, \bm A = \bm A^T \rightarrow \bm H\bm A\bm H$对称}
            \item {$\|x\|_2 = \|y\|_2 \Rightarrow \exists \bm H, \bm Hx = y$,可以将任意一个向量约化到坐标轴上,其中$u = x - y,\bm H = \bm I  - 2\frac{uu^T}{(u,u)}$}
            \item {Givens变换:两轴上的一个旋转变换,可以将任意两轴上$(x_i,x_j)$旋转到$(\sqrt{x_i^2+x_j^2},0)$,$\begin{pmatrix} \cos\theta & \sin\theta \\ -\sin\theta & \cos\theta\end{pmatrix}$}
            \item {QR分解定理和实舒尔分解:$\bm A = \bm Q\bm R,\bm Q^T\bm A\bm Q = \bm U_{\bm R}$其中一阶$\bm R_{ii}$是一阶特征值,二阶$\bm R_{ii}$特征值$\bm A$共轭特征值}
            \item {QR方法:迭代过程:$\bm A_k = \bm Q_k\bm R_k,\bm A_{k+1} = \bm R_k\bm Q_k \Rightarrow \bm A_k\bm Q_k = \bm Q_k\bm A_{k+1}\Rightarrow \bm A_{k+1} = \bm Q_k^T\bm A_k\bm Q_k$ }
            \item {收敛:若$\bm A$特征值满足$|\lambda_1| >|\lambda_2| >\cdots>|\lambda_n|$,可以对角化成标准型$\bm A = \bm X \bm D \bm X^{-1}$,$\bm X$可以LU分解,则$\{\bm A_k\}$本质上收敛于上三角矩阵,$\lim_{k\to\infty}a_{ii}^k = \lambda_i,a_{ij,i\ne j} = 0$}
            \item {原点位移法:$\bm A_k -s_k\bm I= \bm Q_k\bm R_k,\bm A_{k+1} = \bm R_k\bm Q_k + s_k\bm I$,调整每一步的$s_k$使得收敛率$|\frac{\lambda_n - s_k}{\lambda_{n-1}-s_k}|$尽可能小}
            \item {给定原点位移为一个特征值$\mu$的上海森伯格矩阵(带一个下副对角线的上三角矩阵)进行一次原点位移QR分解之后$h_{n,n-1} = 0,h_{nn} = \mu$(单步QR方法)}
            }
\section{补充材料}
    \subsection{Newton法误差的精确分析}
    原则上$e_{n+1} \le  \psi'(x)e_n + \Delta $,在精度足够高的情况下$\psi'(x) \approx \psi (x^*) = 0,e_{n+1} = \Delta$。然而手算不能这么算。
    非线性方程求根,$x_{n+1} - x^* = \psi(x_n) - \psi(x^*) \Rightarrow e_{n+1} = \psi'(x^*)e_n + \cdots + \frac{\psi^{(m)}e^m_n}{m!}$\\ 
    对于Newton法,变为$e_{n+1} = \frac{\psi''(\xi)e_n^2}2$,选取初始区间$[x^*-\delta,x^*+\delta]$,可以得到$M_2$,则$\frac{M_2}2|e_{n+1}| \le {\frac{M_2|e_n|}2} ^2 \le [\frac{M_2}2|e_0|]^{2^{n+1}}$这里$n$不足够大因此$e_{n+1} \ne 0$。当$\frac{M_2}2|e_0|<1$,相当于$|\psi'| < 1$的条件。\\
    \subsection{Romberg 方法相关补充}
    $T_m(h)$为某个复化算法分割$h$长度的效果$m$为其阶数,Richardson消去得到前面写的式子$T_m(h) = \frac{4^m}{4^m-1}T_{m-1}(\frac h2) - \frac {T_{m-1}(h)}{4^m-1}$,分析表如下所示,阶次越高(向右)区间误差越小,分割越小(向下)区间长度越小。(牛顿均差的表跟这个的结构是一样的)
    \subsection{利用积分公式求解微分公式}
    $\int_{x_{k-1}}^{x_{k+1}}f' = f(x_{k+1}) - f(x_{k-1})$,利用机械求积公式写成$I = \sum A_kf'_k$的形式反向求线性方程组,如果使用中矩形公式则可以直接求取,效果等同于中点公式。中间差商公式可以采用Richardson外推加速进行加速,$G(h) = \frac{f(x+h) - f(x-h)}{2h} = f'(x) + \alpha h^2 + \mathcal O(h^4)$,得到$G_1(h) = \frac43G_0(\frac h2) -\frac13G_0(h)$\\
    \subsection{改进欧拉法和欧拉法的累积误差分析}
    方法单步误差:$\frac {h^2}2y{(2)}(\xi)$。记$M \ge |\frac{\partial f}{\partial y}(x,y)|,L \ge y^{(2)}(\xi)$,方法误差传播:$\Delta_{n+1} = (1 + h\frac{\partial f}{\partial y}(x_n,y_n))\Delta_n + \frac {h^2}2y^{(2)}(\xi)$ ,配参数可以得到$\Delta_{n+1} + \frac {Lh^2}{2hM} \le (1 + hM)(\Delta_{n} + \frac {Lh^2}{2hM}) \le (1 + hM)^{n+1}\frac {Lh^2}{2hM}$\\
    对欧拉法,如果考虑存储误差为$\Delta = \frac{10^{-m}}2$,存储误差同上得到$\delta_{n+1}\le (1+hM)\delta_n + \Delta$。变换得到$\delta_{n+1} + \frac{10^{-m}}{2hM} \le (1 +hM)(\delta_n+\frac{10^{-m}}{2hM}) \le (1 + hM)^{n+1}\frac{10^{-m}}{2hM}$。
    上述方法误差和存储误差没有联合分析,联合分析相对更加复杂。\\
    改进欧拉法:方法累计误差$\overline{\Delta_{n+1}} \le (1+hM)\Delta_n + \frac{Lh^2}2,\Delta_{n+1} \le \Delta_n + \frac{hM\Delta_n}2 + \frac {hM\overline{\Delta_{n+1}}}2 + \frac{Th^3}{12}$,其中$T \ge y^{(3)}$\\
    $\Rightarrow \Delta_{n+1} \le (1 + hM + \frac{h^2M^2}2)\Delta_n + (\frac{LM}4+\frac T{12})h^3$\\
    舍入累计误差:$\overline{\delta_{n+1}} \le (1 + hM)\delta_n + \Delta,\delta_{n+1} \le \delta_n + \frac {hM\delta_n}2 + \frac{hM\delta_{n+1}}2 + \Delta$\\
    $\Rightarrow \delta_{n+1} \le (1 + hM + \frac{h^2M^2}2)\delta_n + (1 +\frac{hM}2\Delta)$\\
    \subsection{直接LU分解法}
    \List{
        \item {$u_{1i} = a_{1i}, l_{i1} = a_{i1} / u_{11}; u_{ri} = a_{ri} - \sum_{k=1}^{r-1} l_{rk}u_{ki}$}
        \item {$l_{ir} = (a_{ir} - \sum_{k=1}^{r-1} l_{ik}u_{kr})/ u_{rr}$}
        \item {$y_1 = b_1, y_i = b_i - \sum_{k=1}^{i-1} l_{ik}y_k$}
        \item {$x_n = y_n / u_{nn},x_i = (y_i-\sum_{k=i+1}^nu_{ik}x_k) / u_{ii}$}
    }
    \subsection{追赶法:求解三对角矩阵}
    \List{
        \item {要求$|b_1| >|c_1| >0,|b_n| > |a_n| > 0, |b_i| \ge |a_i| + |c_i|$}
        \item {追:$\beta_1 = c_1/b_1, \beta_i = c_i/(b_i-a_i\beta_{i-1})$}
        \item {追:$y_1 = f_1 / b_1, y_i = (f_i - a_iy_{i-1})/(b_i - a_i\beta_{i-1})$}
        \item {赶:$x_n = y_n,x_i = y_i -\beta_ix_{i+1}$}
        \item {$\begin{pmatrix} b_1 & c_1 & \\ a_1 & b_2 & c_2 \\ & a_2 & b_3 \end{pmatrix} = \begin{pmatrix} \alpha_1 & & \\ r_2 & \alpha_2 & \\ & r_3 & \alpha_3 \end{pmatrix} \begin{pmatrix} 1 & \beta_1 & \\ & 1 & \beta_2 \\ & & 1\end{pmatrix}$}
    }

\end{multicols}
\end{document}